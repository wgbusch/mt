\subsection{Introducción}

En \textit{aprendizaje por computadora} uno de los problemas más fundamentales es el de \textit{clusterizar}.
Clusterizar significa agrupar un conjunto de objetos en diferentes subconjuntos por alguna
nocion de similaridad intraconjunto (o disimiliradad interconjunto). Para esta aplicación, tenemos
una nube de puntos en un espacio métrico, y nos gustaría realizar divisiones (\textit{clusters}) que
agrupen a los conjuntos de puntos que se encuentran juntos según la distancia Euclidiana, no deberían poder quedar nodos sin un conjunto y además los conjuntos deben formar una partición.

\vskip 8pt

Para resolver el problema de la \textit{clusterización} se modelarán los puntos con grafos, donde cada cluster corresponderá a una componente conexa. Como muestra \textit{Charles Zahn}, los árboles generadores mínimos "preservan" la clusterización, serán un recurso muy importante a lo largo de este trabajo, pues la idea principal detrás de este proyecto es utilizar un árbol generador mínimo para obtener un camino mínimo a través de todos los puntos del eje cartesiano y luego recortar los ejes correspondientes que separen un cluster de otro, de esta manera generando cada componente conexa relacionada con cada cluster, donde el criterio que decida si un eje debe ser cortado o no será si el largo de este eje en cuestión es demasiado grande o no en comparación con el resto de los ejes del grafo. Esta es la noción de \textit{eje inconsistente} que presenta \textit{Zahn}

\vskip 8pt

Sin embargo, al no haber definición formal de \textit{cluster} la validez de una disposición depende del uso que se le dé. Se estudiarán diversos criterios para armar \textit{clusters} que serán \textit{coherentes}. Motivados por esto, intentaremos analizar y experimentar sobre las posibles soluciones para cada problema de este tipo y cómo fluctuan en función de los algoritmos utilizados y sus parámetros de entrada.

\subsubsection{Sobre el AGM}
Como se menciono en la subsección anterior, el \textit{árbol generador mínimo} de un grafo \textit{conexo} preservar los clusters. Se modela la lista de puntos a \textit{clusterizar} como un grafo completo, donde cada punto está conectado con todos y cada eje entre dos nodos $u, v$ tiene un peso asignado que denota la distancia Euclidiana entre el nodo $u$ y el nodo $v$. Dado este grafo $G$, se procede a generar un árbol generador mínimo $T$ asociado a $G$. De esta manera nos quedaremos sólo con las distancias más cortas entre los nodos de nuestro problema de tal manera que todos los nodos permanezcan conectados a través de un camino. Ya existen varios algoritmos que resuelven el problema de hallar el $AGM$ en tiempo razonable, Prim y Kruskal, y ambos seran utilizados para darlo. 
Una vez calculado el árbol generador mínimo $T$ de $G$, la idea principal para la construcción de una \textit{clusterización válida} de una lista de puntos consiste en \textbf{recortar los ejes inconsistentes} del árbol $T$, es decir, remover aquellos ejes que dispongan de un peso atípicamente alto en comparación con el de los ejes que lo rodean. Realizar esta tarea no es un paso trivial y el algoritmo encargado del recorte de estos ejes sobresalientes funciona con una heurística que depende de tres parámetros regulables.
