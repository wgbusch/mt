\subsection{Introducción a la problemática}
% descripcion del problema
% algoritmos propuestos para resolverlo
% Bellman-Ford
% Floyd-Warshall

\subsubsection{Noción de arbitraje}
En finanzas, se denomina arbitraje a la práctica de comprar y vender un recurso de manera simultanea para generar una ganancia, aprovechandose de los desbalances de los precios en diferentes mercados. Supongamos que existe un recurso R disponible para la compra y venta en dos distintos mercados, $M_{1}$ y $M_{2}$.
 \par
En $M_{1}$, se puede comprar R por 42 pesos y vender por 41.8.
 \par
En $M_{2}$, se puede comprar R por 42,5 pesos y vender por 40,9.
 \par
Comprar en el segundo mercado para venderlo en el primero, o viceversa, no genera ganancia. Sin embargo, si el recurso se valoriza y $M_{1}$ se entera de este suceso: actualizará su cotización. De esta forma, ahora $M_{1}$ compra por 42,8 pesos el recurso y lo vende por 43,5. Hay oportunidad de arbitraje ya que $M_{2}$ no actualizó sus precios: por cada unidad comprada en $M_{2}$ y vendida en $M_{1}$ se obtiene 43,5-42,5 = 1 peso. 

\subsubsection{Arbitraje de divisas}
El problema a resolver en este trabajo es analizar si existe oportunidad de arbitraje entre varias divisas (pesos, dolares, euros, etc). Para ello, contamos con la tasa de cambio para cada divisa.

La entrada consistirá de un primer entero $n$ que corresponde a la cantidad de divisas a tener en cuenta. Luego, habrá $n$ líneas. Cada una de ellas tendrá $n$ numeros reales $c_{i,j}$, representando el multiplicador que se debe aplicar a una unidad de la divisa $i$ al cambiar a la divisa $j$. Si el arbitraje existe, la salida debe ser un posible ciclo de divisas, donde cada numero representará desde la divisa $0$ a la $n-1$.

A continuación se presentan dos ejemplos, uno donde existe arbitraje y otro donde no.
\begin{table}[!htb]
\begin{minipage}{.5\linewidth}
\begin{center}
	\begin{tabular}{ |l l l l|l| } 
		\hline
		&Entrada& de& ejemplo & Posible salida \\
		\hline
		4 &&&& 3 2 1 3 \\
		1& 0.1& 5& 0.125 & \\ 
		10& 1& 0.5& 0.3  & \\ 
		0.2& 2& 1& 2 & \\ 
		8& 3& 0.5& 1 & \\ 
		\hline
	\end{tabular}
\end{center}
\end{minipage}
\begin{minipage}{.5\linewidth}
\begin{center}
	\begin{tabular}{ |l l l l|l| } 
		\hline
		&Entrada& de& ejemplo & Salida\\
		\hline
		4 &&&& NO \\
		1& 0.1& 0.05& 0.125 & \\ 
		0.1& 1& 0.5& 0.3  & \\ 
		0.2& 0.02& 1& 0.2 & \\ 
		0.8& 0.3& 0.5& 1 & \\ 
		\hline
	\end{tabular}
\end{center}
\end{minipage}
\end{table}
%En el ejemplo de arriba, el ciclo es $\{2, 3, 1, 2\}$  ya que $c_{23}$ x $c_{31}$ x $c_{12}$ x $c_{21} = 3 > 1$
\par
Por lo tanto, debemos hallar una secuencia de divisas tal que al convertir una unidad de la divisa inicial a través de las monedas, se genere una ganancia. Esto se traduce a encontrar una secuencia S 
\begin{align*}
S =& \{d_a, d_b, d_c, ... , d_k\}\quad \text{tal que} \quad c_{a,b} \: \text{x} \: c_{b,c} \: \text{x ... x} \: c_{k,a} > 1 
\end{align*}
\par
En la entrada de ejemplo con solucion, un posible ciclo es $\{3, 2, 1, 3\}$  ya que $c_{32}$ x $c_{21}$ x $c_{13}$ = 2 x 10 x 5 = 100 $>$ 1.

